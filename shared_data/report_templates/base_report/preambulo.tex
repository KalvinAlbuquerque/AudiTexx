% Load xcolor with 'table' option BEFORE other packages that might load it
\usepackage[table]{xcolor} % Corrected: xcolor loaded once with [table] option.

% Pacotes essenciais
\usepackage[brazil]{babel}
\usepackage[T1]{fontenc}

% Pacotes para citações no estilo ABNT
\usepackage[alf]{abntex2cite}

\usepackage{graphicx}

% Pacotes adicionais (xcolor REMOVIDO desta lista, tikz ainda está aqui, mas xcolor já foi carregado)
\usepackage{csquotes, comment, enumerate, multirow,
    multicol, titlesec, amsmath, amsthm, amsfonts, amssymb, dsfont,
    blindtext, ragged2e, array, enumitem, tikz, bbding, pifont, wasysym,
    titling, longtable, fancyhdr, url, float, placeins, xurl}

% Configuração de margens
\usepackage[lmargin=2cm, rmargin=2cm, tmargin=2cm, bmargin=2.5cm]{geometry}

% Configuração do cabeçalho com logo
\usepackage{fancyhdr}
\pagestyle{fancy}
\fancyhf{}
\renewcommand{\headrulewidth}{0pt} % Remove a linha no cabeçalho
\rhead{\includegraphics[width=2cm]{./assets/logocogel.jpg}} % Corrigido: Removido \transparent

% Configuração do hyperref (deve ser o último pacote)
\usepackage{hyperref}
\hypersetup{
    colorlinks=true,
    linkcolor=blue, % Cor dos links internos (tabelas, figuras, etc)
    urlcolor=blue,  % Cor dos links externos
    citecolor=green % Cor das citações
}

\usepackage{helvet}
\renewcommand{\familydefault}{\sfdefault}

% Configuração da bibliografia (ABNT)
\bibliographystyle{abntex2-alf}